\documentclass[12pt]{amsart}

\usepackage{amsmath, amsthm, amssymb, amsfonts, fullpage, nopageno, skull}

\usepackage{graphicx}

\setlength{\parindent}{0in}

\def\CC{{\mathbb C}}
\def\FF{{\mathbb F}}
\def\QQ{{\mathbb Q}}
\def\RR{{\mathbb R}}
\def\ZZ{{\mathbb Z}}
\def\NN{{\mathbb N}}

\def\Ecal{{\mathcal E}}
\def\Fcal{{\mathcal F}}
\def\Ocal{{\mathcal O}}

\begin{document}

{\bf Winter 2021 MTH 440/540 Homework 1}

{\bf Wyatt Whiting}

\medskip


{\bf 1.}  For each pair $a$ and $b$, find $q,r\in\ZZ$ with $a=qb+r$ and $0\leq r<b$.
\begin{itemize}
\item[{\bf (a)}] $a=64$ and $b=11$

$64 = 11 \cdot 5 + 9 \implies q = 11, r = 9$

\item[{\bf (b)}] $a=-50$ and $b=7$

$-50 = 7 \cdot -8 + 6 \implies q = -8, r = 6$

\item[{\bf (c)}] $a=91$ and $b=13$

$91 = 13 \cdot 7 + 0 \implies q = 7, r = 0$

\item[{\bf (d)}] $a=11$ and $b=15$

$11 = 15 \cdot 0 + 11 \implies q = 0, r = 11$

\end{itemize}

\medskip

{\bf 2.}  Prove that $6\mid n^3-n$ for all integers $n$.  [Hint: First check it for $0\leq n\leq 5$. Then reduce to this case by dividing $n$ by $6$ with remainder.]

\bigskip

We begin by checking case $n = 1$. We then have $n^3 - n = 1^3 - 1 = 0$, and cleary $6 | 0$ because $0 = 6 \cdot 0 + 0$.

We now perform an inductive step. Assume $6 | n^3 - n$ for some $n \in \NN$. Now, we may rearrange the expression as such:

\[(n + 1)^3 - (n +1) = n^3 + 3n^2 + 3n + 1 - n - 1 = (n^3 - n) + 3n(n + 1)\]

We know the product $n(n + 1)$ must be even since either $n$ or $n + 1$ must be even, so $2 | n(n + 1)$. Additionally, 3 clearly divides $3n(n + 1)$ since $3n(n + 1) = 3 \cdot (n (n + 1))$. Therefore, since both 2 and 3 are factors of $3n(n + 1)$, then $6 | 3n(n + 1)$. We then have $6 | n^3 - n$ by the induction hypothesis and $6 | 3n(n + 1)$. Therefore, 6 must also divide the sum $(n^3 - n) + 3n(n + 1) = (n + 1)^3 - (n + 1)$. We have therefore shown that $6 | n^3 - n \implies 6 | (n + 1)^3 - (n + 1)$. By the principle of induction, $6 | n^3 - n$ for all $ n \in \NN$.  

Now, let $a, b \in \ZZ$ be such that $a | b \implies b = ka$ for some $k \in \ZZ$. We can then also say that $a | -b$, as we can choose $m = -k$ so that $-b = ma = -ka$. We have already demonstrated that $6 | n^3 - n$ for all positive integers n. $n^3 - n$ is an odd function, so if $n^3 - n = p$, then $(-n)^3 - (-n) = -p$. Therefore, if $6 | n^3 - n \implies 6 | p$, then it must also be the case that $6 | -p \implies 6 | (-n)^3 - (-n)$ for all $n \in \NN$.

Now we just need to check the case for $n = 0$, which is trivially true, as $0^3 - 0 = 0 = 0 \cdot 6$, so $6 | 0$. 

We have then shown that $6 | n^3 - n$ and $6 | (-n)^3 - (-n)$ for all $n \in \NN$, which together constitute all non-zero integers. Together with the case of $n = 0$, we have proven that $6 | n^3 - n$ for all integers. $\skull$

\bigskip

{\bf 3.}  The first few primes of the form $6x+5$ (for $x\in\ZZ$) are $5,11,17,23,29,41,47,53,59,71,\dots$.  In this problem you will show that there are infinitely many primes of the form $6x+5$.
\begin{itemize}
\item[{\bf (a)}] Let $p$ be a prime number which is not $2$ or $3$.  Show that when $p$ is divided by $6$, the remainder is either $1$ or $5$.

\medskip

Let $p$ be an arbitrary prime greater than 3. If we divide $p$ by 6 with remainder, we can express it in the form $p = 6 \cdot q + r$ for some $q \in \ZZ$ with $0 \leq r < 6$. The remainder $r$ cannot be 0, 2, or 4, since this would make $6 \cdot q + r$ an even number, which would contradict our assumption that $p$ is prime. Likewise, the remainder cannot be 3, because $6 \cdot q + 3 = 3(2\cdot q + 1)$ has 3 as a factor, which would contradict our assumption of $p$'s primality. Since $r$ cannot be 0, 2, 3, or 4, it must be the case that either $r = 1$ or $r = 5$.

\medskip

\item[{\bf (b)}] Show that the product of two numbers of the form $6x+1$ is also of the form $6x+1$.

\medskip

Let $6x + 1$ and $6y+1$ be such that $x,y\in\ZZ$. If we take the product of these two terms, we get 

\[(6x + 1)(6y + 1) = 36xy + 6x + 6y + 1 = 6(6xy+x+y)+1. \]

The integers are closed both under multiplication and addition, so the term $6xy + x + y$ must also be an integer. Therefore, the product of two numbers of the form $(6x+1)$ where $x\in\ZZ$ must also be of the form $(6x + 1)$ for some $x \in \ZZ$.

\medskip

\item[{\bf (c)}] Show that if $k$ is a positive integer, then $6k+5$ has a prime factor $p$ of the form $p=6x+5$.   

\medskip

We begin by noting $6k + 5$ will always be odd, so 2 cannot be a factor. Additionally $6k + 5 = 3(2k + 1) + 2$, so 3 does not divide $6k+5$ either and cannot be a factor. By the proof in section 3a, any prime factor must either be of the form $6x + 1$ or $6x + 5$. If $6k + 5$ has a prime factor of the form $6x + 5$, then we can take $x = k$ to form that factor. We now consider the case if $6k + 5$ has a prime factor of the form $6k + 1$ and not one in the from $6k+5$. Since now $6k + 5$ must be composite, it must have at least two prime factors, both of which are in the form $6x + 1$. But from the proof in 3b, the product of these two factors would also be of the form $6x + 1$, which cannot be the case since we are only considered numbers of the form $6k + 5$. Therefore, it cannot have any prime factors of the forms $6x +1$, so $6k + 5$ must have a prime factor of the form $6x + 5$. 

\medskip

\item[{\bf (d)}] Modify Euclid's proof to show\footnote{It is also true that there are infinitely many primes of the form $6x+1$, but this is harder to show.} that there are infinitely many primes of the form $6x+5$. 

\medskip

Suppose there are only finitely many primes of the form $6x + 5$, enumerated in the set $\{p_1, p_2,\cdots,p_n\}$. Consider the value $q = 6p_1 p_2 \cdots p_3 - 1 = 6(p_1p_2\cdots p_n - 1) + 5$. By construction, no $p_i$ divides $q$. By section 3a, any prime divisors of $q$ must have the form $6x + 1$ or $6x + 5$. By section 3c, $q$ must have at least one factor of the form $6x + 5$. If this were not the case, all prime factors of $q$ would be in the form $6x + 1$, and by section 3b $q$ would also be of the from $6x + 1$. However, the result that $q$ must have at least one prime factor in the form $6x + 5$ contradicts the construction of $q$, which guarantees no prime of the form $6x + 5$ may factor it. Therefore, our assumption of the finiteness of primes in the from $6k + 5$ must be incorrect, so we may conclude there are infinitely many primes of the form $6x + 5$. $\skull$

\medskip


\end{itemize}

{\bf 4.} In class we proved this theorem: 

\medskip

{\bf Theorem:} Let $a$ and $b$ be integers (not both zero).  Then $\gcd(a,b)=1$ if and only if $ax+by=1$ for some integers $x$ and $y$.  

\medskip

Using this theorem, prove the following statements. (Do not use the Fundamental Theorem of Arithmetic.)  Here $a,b,c$ are nonzero integers.
\begin{itemize}
	\item[{\bf (a)}]  Show that if $\gcd(a,b)=1$ and $a\mid c$ and $b\mid c$, then $ab\mid c$.
	
	\medskip
	
	We know there exist some $x, y$ such that $ax + by = 1 \implies cax + cby = c$. Since $a|c \implies c = na$ and $b | c \implies c = mb$ for some $m$ and $n$, we can then say that $mbax + naby = c \implies ab(mx + ny) = c \implies ab | c$ because $(mx+ny)\in\ZZ$ is an integer. 	
	
	\medskip	
	
	
	\item[{\bf (b)}]  Show that if $\gcd(a,b)=1$ and $a\mid bc$, then $a\mid c$. 
	\medskip
	
	We know there exist some $x, y$ such that $ax + by = 1 \implies cax + cby = c$. We also have $a | bc \implies bc = na$ for some $n$. Thus, $cax + nay = c \implies a(cx + ny) = c$, and because $(cx + ny)$ is an integer it must be the case that $a | c$.
	
	\medskip	
	\item[{\bf (c)}]  Show that if $p$ is a prime number and $p\mid ab$, then either $p\mid a$ or $p\mid b$. 
	\medskip
	
	Since $p$ is prime, we have $\gcd(p, a) = 1 \implies px + ay = 1$ and $\gcd(p, b) = ps + bt = 1$ for some integers $x, y, s, t$. We also have $p | ab \implies p=n\cdot ab$ for some integer $n$. We can multiply both sides of $px + ay = 1$ by $b$ to obtain $bpx + bay = b$. The first term on the left is divisible by $p$, and by assumption $ab = ba$ is divisible by $p$, so the sum $bpx + bay = b$ is divisible by $p$, and therefore $p | b$. 
	
	We may do the same starting with the other equation: $ps + bt = 1 \implies psa + abt = a$, and we know $p | psa$ and $p | abt$ by the same reasoning above, so it must be the case that $p | (psa + abt) \implies p | a$.
	
	We then conclude that either $p|a$ or $p|b$.
	 
	\medskip	
	
	\item[{\bf (d)}]  Show that $\gcd(6x+5,5x+4)=1$ for all $x\in\ZZ$. 
	
	\medskip
	
	To demonstrate this, I will carry out Euclid's algorithm. Let $x\in\ZZ$ be arbitrary.
	
	$6x + 5=1\cdot(5x+4)+(x + 1)$
	
	$5x + 4 = 4\cdot(x+1) + x$
	
	$x+1=1\cdot(x)+1$
	
	$x=x\cdot(1)+0$
	
	
	In the step before obtaining a remainder 0, we have remainder 1, which is then the greatest common divisor. We can then see that $\gcd(6x+5,5x+4)=1 \; \forall x\in\ZZ$.
	
	\medskip	
	
	
\end{itemize}

\medskip

{\bf 5.}  For each pair $a$ and $b$, use Euclid's algorithm to find $d=\gcd(a,b)$ and find $x,y\in\ZZ$ such that $ax+by=d$.
\begin{itemize}
\item[{\bf (a)}] $a=-23$ and $b=16$
\medskip
$-23 = -2 \cdot 16 + 9$

$16 = 1\cdot 9 + 7$

$9 = 1 \cdot 7 + 2$

$7 = 3 \cdot 2 + 1$

$2 = 2\cdot 1 + 0$

So $\gcd(-23, 16) = 1$. If we take $x=9, y = 13$, we see that $-23\cdot9 + 16\cdot 13 = 1$.
\medskip
\item[{\bf (b)}] $a=111$ and $b=442$
\medskip

$442 = 3 \cdot 111 + 109$

$111 = 1 \cdot 109 + 2$

$109 = 54 \cdot 2 + 1$

$2 = 2 \cdot 1 + 0$

So $\gcd(111, 442) = 1$. If we take $x=223, y = -56$, we see that $111\cdot 223 + (-56)\cdot 442 = 1$.

\medskip

\end{itemize}

\medskip

{\bf 6.}  Use Sage's {\tt is\_prime()} command to find the answers to the following questions.  You may wish to use (suitably modified) versions of the programs on Homework 0 (the Sage warm-up).  
 \smallskip
\begin{itemize}
\item[{\bf (i)}] How many three-digit primes are there?

\medskip

\begin{verbatim}
n = 0
for i in range(100, 1000): 
    if is_prime(i): 
        n = n + 1
print(n)

>143
\end{verbatim}
There are 143 3-digit primes
\medskip

\item[{\bf (ii)}] List the three smallest ten-digit primes. 
\medskip
\begin{verbatim}
n = 0
p = 1000000001 # start with first ten-digit odd number
while n < 3:
    if is_prime(p): 
        print(p)
        n = n + 1
    p = p + 2 # only check odd numbers since no evens will be prime
    
> 1000000007
> 1000000009
> 1000000021
\end{verbatim}
The three numbers above are the three smallest 10-digit primes
\medskip 
\item[{\bf (iii)}] Some primes\footnote{Nobody knows whether there are infinitely many primes of the form $n^2+1$.} have the form $n^2+1$, such as $1^2+1=2$, $2^2+1=5$, and $4^2+1=17$.  List all four-digit primes of the form $n^2+1$.  How many are there?
\medskip
\begin{verbatim}
for n in range(32, 99): # n^2 + 1 guaranteed to be 4-digit
    if is_prime(n*n + 1):
        print(n*n + 1)
        
> 1297
> 1601
> 2917
> 3137
> 4357
> 5477
> 7057
> 8101
> 8837
\end{verbatim}
There are 9 4-digit primes of the form $n^2 + 1$.


\end{itemize}   

\medskip


\end{document}